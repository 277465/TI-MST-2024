%Jakub Klimek Grupa 2, 277441
%
% Wiele się na tym nie nauczy, ale formalnie dobrze.
%
% 10+2=12 pkt.
%
\documentclass{article}
\usepackage{amsmath}
\usepackage{geometry}
\usepackage{polski}
\geometry{
    a4paper,
    left=3cm,
    right=3cm,
    top=2cm,
    bottom=2cm,
}

\begin{document}
 Dzielimy obie strony tożsamości znowu przez \(x-1\):
\[
-2x + 1 \equiv E x + F + (G x + H)(x^2 + 1) \equiv E x + F + G x^3 + G x + H x^2 + H,
\]
skąd \(G = 0\), \(H = 0\), \(E = -2\), \(F = 1\).

Mamy więc
\[
\frac{x^6 - 6x^5 + 10x^4 - 17x^3 + 8x^2 - 5x + 1}{(x^4 + 2x^2 + 1)(x^4 - 3x^3 + 3x^2 - x)} \equiv -\frac{1}{x} - \frac{2}{(x - 1)^3} + \frac{1}{x - 1} + \frac{-2x + 1}{(x^2 + 1)^2}.
\]
Całkując obie strony tożsamości otrzymujemy
\begin{align*}
I &= -\int \frac{dx}{x} - 2 \int \frac{dx}{(x - 1)^3} + \int \frac{dx}{x - 1} + \int \frac{-2x + 1}{(x^2 + 1)^2} \, dx= \\
&= -\ln |x| + \frac{1}{(x - 1)^2} + \ln |x - 1| - \int \frac{2x \, dx}{(x^2 + 1)^2} + \int \frac{dx}{(x^2 + 1)^2}.
\end{align*}

\end{document}
